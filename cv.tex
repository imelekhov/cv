% Original author of this template:
% Trey Hunner (http://www.treyhunner.com/)

%----------------------------------------------------------------------------------------
%	PACKAGES AND OTHER DOCUMENT CONFIGURATIONS
%----------------------------------------------------------------------------------------

\documentclass{resume} % Use the custom resume.cls style

\usepackage[left=0.75in,top=0.6in,right=0.75in,bottom=0.6in]{geometry} % Document margins

\name{Iaroslav Melekhov} % name
\address{Paalikatu 13a, 21 \\ Oulu, Finland 90520} % address
\address{(040)~$\cdot$~245~$\cdot$~5752 \\ imelekho@ee.oulu.fi} % phone number and email

\begin{document}

%----------------------------------------------------------------------------------------
%	EDUCATION SECTION
%----------------------------------------------------------------------------------------

\begin{rSection}{Education}

{\bf University of Oulu} \hfill {\em June 2015 - Present}  \\ 
Ph.D. Computer Science \\
{\it Thesis advisor}: Juho Kannala \\
Courses: Deep Convolutional Neural networks \smallskip

{\bf St.Petersburg Electrotechnical University "LETI"} \hfill {\em April 2014} \\ 
Ph.D. in location and radar navigation \\
{\it Thesis advisor}: Vladimir K. Orlov \\
{\it Theme}: "Analysis and development algorithms of joint information processing in relative navigation" \smallskip

{\bf St.Petersburg Electrotechnical University "LETI"} \hfill {\em June 2010} \\ 
M.S. in  Radio Engineering \\
{\it Thesis advisor}: Alexander B. Sergienko \\
{\it Theme}: "Non-coherent digital modulation classification in fading channels" \\
Overall GPA: 4.95 (5.0) \smallskip

{\bf St.Petersburg Electrotechnical University "LETI"} \hfill {\em June 2008} \\ 
B.S. in  Radio Engineering \\
{\it Thesis advisor}: Denis I. Vezhenkov \\
{\it Theme}: "The review of methods of classification OFDM" \\
Overall GPA: 4.89 (5.0) \smallskip

\end{rSection}

%----------------------------------------------------------------------------------------
%	WORK EXPERIENCE SECTION
%----------------------------------------------------------------------------------------

\begin{rSection}{Experience}

\begin{rSubsection}{Research and Development Center Protei}{April 2014 - December 2014}{Software researcher \& developer}{St.Petersburg, Russia}
\item Redesigned a traffic management platform with deep packet inspection (DPI) capabilities;
\item Implemented leaky (token) bucket algorithms to control peak and average data rate.
\end{rSubsection}

%------------------------------------------------

\begin{rSubsection}{JSC “VNIIRA”}{September 2011 - April 2014}{Software developer}{St.Petersburg, Russia}
\item Aircraft trajectory prediction module on the basis of the current flight state, environmental conditions and AC characteristics was developed from scratch.
\end{rSubsection}

%------------------------------------------------

\begin{rSubsection}{JSC “Vector”}{September 2009 - April 2011}{Software developer}{St.Petersburg, Russia}
\item Developed DQPSK, GMSK demodulator;
\item Implemented Viterbi equalizer using MATLAB/C++.
\end{rSubsection}

%------------------------------------------------

\begin{rSubsection}{St. Petersburg Electrotechnical University “LETI”}{September 2006 - July 2009}{Research intern}{St.Petersburg, Russia}
\item Processing of OFDM signals, methods of synchronization;
\item Non-coherent classification algorithms of digital signals in fading channels were
implemented.
\end{rSubsection}

\end{rSection}
\bigskip

%----------------------------------------------------------------------------------------
%	PUBLICATIONS SECTION
%----------------------------------------------------------------------------------------

\begin{rSection}{Publications}
\begin{itemize}
\item \textbf{Relative coordinates estimation under correlated measurement errors} (Russian)\\
ISSN 2071-8985 "Izvestiya SPbGETU LETI"\\
\underline{Y.Melekhov}, V.Orlov
\item \textbf{Comparative analysis of multi-sensor fusion algorithms of relative distance estimation} (Russian) ISSN 2071-8985 "Izvestiya SPbGETU LETI"\\
\underline{Y.Melekhov}, V.Orlov
\item \textbf{Performance of multi-sensor fusion algorithms of relative distance estimations with missing measurements} (Russian) ISSN 2071-8985 "Izvestiya SPbGETU LETI"\\
\underline{Y.Melekhov}, V.Orlov
\item \textbf{Output distance estimation in relative navigation of air vehicles for non linear measurements of coordinates} (Russian) ISSN 2071-8985 "Izvestiya SPbGETU LETI"\\
\underline{Y.Melekhov}, V.Orlov
\end{itemize}

\end{rSection}

%----------------------------------------------------------------------------------------
%	TECHNICAL STRENGTHS SECTION
%----------------------------------------------------------------------------------------

\begin{rSection}{Technical Strengths}

\begin{tabular}{ @{} >{\bfseries}l @{\hspace{6ex}} l }
Programming Languages & C/C++, Matlab, Java, Python \\
Libraries \& Frameworks & stl, boost, Caffe, Qt \\
Concepts & OOA/OOD (Design Patterns), Multithreading/Multiprocessing\\
Systems & Unix (Centos), Arch, OpenSUSE \\
Publishing & LATEX, TEX
\end{tabular}

\end{rSection}

%----------------------------------------------------------------------------------------
%	HONORS AND AWARDS SECTION
%----------------------------------------------------------------------------------------

\begin{rSection}{Honors and awards}
\begin{itemize}
\item {International Computer Vision Summer School (ICVSS), statement of accomplishment July, 2015} \\
\end{itemize}
\begin{itemize}
\item {Internship at Dresden University of Technology, 2010} \\
\end{itemize}

\end{rSection}

%----------------------------------------------------------------------------------------
%	LANGUAGES
%----------------------------------------------------------------------------------------

\begin{rSection}{Languages}
\begin{itemize}
\item {Fluent written and verbal English}
\item {Basic knowledge of German}
\item {Fluent in Russian}
\end{itemize}

\end{rSection}


%----------------------------------------------------------------------------------------

\end{document}